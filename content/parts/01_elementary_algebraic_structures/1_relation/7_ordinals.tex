\section{Ordinals and naturals} \label{ordinals-and-naturals}

In this section, we detail some properties of partial orders that are relevant to constructing the set of natural numbers, which we will build in this section.
\\

\begin{definition} \label{min-max-defs}
    Let $(P,\le)$ be a poset and $T \subseteq P$ be a subset. An element $s \in T$ is said to be
    \\

    \begin{itemize}
        \item[(i)] \textbf{minimal in $T$} if there are no elements in $T$ smaller than it.
        \[
            \forall t \in T, \quad t \le s \implies t = s.
        \]

        \item[(ii)] \textbf{maximal in $T$} if there are no elements in $T$ greater than it.
        \[
            \forall t \in T, \quad t \ge s \implies t = s.    
        \]

        \item[(iii)] \textbf{a minimum of $T$} if every other element of $T$ is greater than it. 
        \[
            \forall t \in T, \quad s \le t.
        \]
        If such a minimum exists, since $P$ is a poset, it is unique ($s_1 \le s_2$ and $s_2 \le s_1$ imply $s_1 = s_2$ by antisymmetry). We denote it by $\min T$. 
        \\

        \item[(iv)] \textbf{a maximum of $T$} if every other element of $T$ is smaller than it. 
        \[
            \forall t \in T, \quad t \le s.    
        \]
        If such a maximum exists, since $P$ is a poset, it is unique. We denote it by $\max T$.
        \\

    \end{itemize}
\end{definition}

\begin{example} \label{opposite-poset}
    Let $(P,\le)$ be a poset. Note that since the symbol used for the homogeneous relation $\le \,\, : P \times P \to P$ is the symbol $\le$, a priori, the symbol $\ge$ means nothing here. We define the relation $\ge$ via
    \[
        \ge \,\, \defn \,\, \le^{\top}, \qquad a \ge b \iff b \le a.     
    \]
    This always defines a poset $(P, \ge)$ because antisymmetry and transitivity trivially follow from the same properties of $(P,\le)$. We call it the \textbf{opposite poset} of $P$. (The same definition holds for other symbols, e.g. $(P,>)$ is the opposite poset of $(P, <)$.)
\end{example}

\begin{remark}
    When $(P,<)$ is a strict poset, the expression $a \le b$ stands as a short-hand for $a < b$ or $a=b$. In particular, strict posets can have minima by interpreting Definition \autoref{min-max-defs} with this convention. Similarly, if $(P, \le)$ is a non-strict poset, then $a < b$ is short-hand for $a \le b$ and $a \neq b$. 
\end{remark}

\begin{definition} \label{well-ordered-sets}
    Let $\alpha = (P,<)$ be a strict poset. We say that $\alpha$ is 
    \\

    \begin{itemize}
        \item[(i)] \textbf{well-ordered} if it is linearly ordered and every non-empty subset $S \subseteq P$ has a mininum.
        \\

        \item[(ii)] \textbf{ordinal} if it is well-ordered, and for $p,q \in P$, we have
        \\

            \begin{itemize}
                \item[$\bullet$] $p < q$ if and only if $p \in q$
                \\

                \item[$\bullet$] $p \in P \implies p \subseteq P$.
                \\
            \end{itemize}

        \item[(iii)] \textbf{finite} if $(P, <)$ and the opposite poset $(P, >)$ are both well-ordered. In other words, $\alpha$ is finite if every non-empty subset $T \subseteq P$ admits a minimum and a maximum.
        \\

        \item[(iv)] an \textbf{infinite ordinal} if it is well-ordered but not finite (so that some non-empty subset of it admits no maximum).
    \end{itemize}
\end{definition}

\begin{remark}
    In practice, to show that a poset $(P,<)$ is well-ordered, it suffices to show that every non-empty subset $T \subseteq P$ admits a minimum. For if this is true and $p,q \in P$ are distinct, the subset $\{p,q\} \subseteq P$ has a minimum. If it is $p$, then $p = \min\{p,q\} < q$. If it is $q$, then $q = \min\{p,q\} < p$. This proves $P$ is linearly ordered. However, the proof that every subset $T \subseteq P$ admits a minimum often relies on the case where $T = \{p,q\}$ to be proven first.
\end{remark}

\begin{remark}
    If $(P, <)$ is linearly ordered, then $(P, >)$ is automatically linearly ordered. So to show that a well-ordered poset is finite, it suffices to find a maximum for any non-empty subset $T \subseteq P$.
\end{remark}

\begin{remark}
    If $\alpha = (P,<)$ is an ordinal, then $p < q$ if and only if $p \in q$. So once we know that the set $P$ defines an ordinal, the data of the relation $<$ doesn't add any information. We therefore reserve the right to denote by $\alpha$ both the poset and the underlying set that defines the ordinal, i.e. $\alpha = (\alpha, \in)$.
\end{remark}


\begin{definition} \label{morphisms-of-posets}
    Let $\Phi : (P,\le) \to (Q,\le)$ be a function between two posets, i.e. $\Phi : P \to Q$ is a function. 
    \\

    \begin{itemize}
        \item[(i)] We say that $\Phi$ is \textbf{non-decreasing} if for all $p,p' \in P$, the inequality $p \le p'$ implies $\Phi(p) \le \Phi(p')$. If $(P,<)$ and $(Q,<)$ are strict posets instead, and $p < p'$ implies $\Phi(p) < \Phi(p')$, we say that $\Phi$ is \textbf{increasing}.
        \\

        \item[(ii)] If $\Phi$ is bijective and $\Phi^{-1} : (Q,\le) \to (P,\le)$ is also non-decreasing, then $\Phi$ is said to be a \textbf{poset isomorphism}. When there exists a poset isomorphism between $(P,\le)$ and $(Q,\le)$, we write $(P,\le) \simeq (Q,\le)$.
        \\

    \end{itemize}
\end{definition}

\begin{proposition} \label{extrema-under-poset-morphisms}
    Let $\Phi : (P,\le) \to (Q,\le)$ be a non-decreasing function and $T \subseteq P$ be a subset.
    \\

    \begin{itemize}
        \item[(i)] Suppose $\Phi$ is a poset isomorphism. The element $p \in T$ is minimal (resp. maximal) if and only if $\Phi(p) \in \Phi(T)$ is minimal (resp. maximal).
        \\

        \item[(ii)] Suppose $\Phi$ is surjective. If $T$ admits a minimum (resp. a maximum), then so does $\Phi(T)$ and $\Phi(\min T) = \min \Phi(T)$ (resp. $\Phi(\max T) = \max \Phi(T)$).
        \\

    \end{itemize}
\end{proposition}

\begin{proof}
    \begin{itemize}
        \item[(i)] Let $q \in \Phi(T)$ be such that $q \le \Phi(p)$. It follows that $\Phi^{-1}(q) \le p$, therefore $\Phi^{-1}(q) = p$ and $q = \Phi(p)$. We deduce that $\Phi(p)$ is minimal. Since $\Phi^{-1}$ is also a poset isomorphism, the reverse implication follows from the one we just proved applied to $\Phi^{-1}$. The proof in the case of maximality follows from the proof of minimality applied to the opposite posets $(P, \ge)$ and $(Q, \ge)$. 
        \\

        \item[(ii)] Let $q \in \Phi(T)$, and find $p \in T$ such that $\Phi(p) = q$, which exists because $\Phi$ is surjective. By definition, we have $\min T \le p$, therefore $\Phi(\min T) \le \Phi(p) = q$. It follows that $\Phi(\min T) = \min \Phi(T)$. Again, the proof applied to the posets $(P, \ge)$ and $(Q, \ge)$ proves the case of the maximum. 
        \\

    \end{itemize}
\end{proof}

\begin{remark}
    It is important that the inverse map to $\Phi$ is non-decreasing to be able to carry over the minimum from $P$ to $Q$. Suppose we start with a poset $(P,\le)$ that has a minimal element $p$, and we consider the poset $(P,=)$ where the only pairs of comparable elements are of the form $(p,p)$ for $p \in P$ (one can easily see that this defines a poset). The identity map $\id P : (P,=) \to (P, \le)$ is bijective and vacuously non-decreasing, and furthermore every element of $(P, =)$ is minimal. Clearly not every element of $P$ can be minimal for any poset $p$, so even being bijective and non-decreasing isn't enough to allow $\Phi$ to send minimal elements to minimal elements. 
\end{remark}

\begin{proposition} \label{morphisms-of-linearly-ordered-posets}
    Let $\Phi : (P, <) \to (Q,<)$ be an increasing function between two strict posets, and assume $(P,<)$ is linearly ordered.
    \\

    \begin{itemize}
        \item[(i)] The function $\Phi$ is injective.
        \\

        \item[(ii)] If $\Phi$ is surjective, then $\Phi$ is a poset isomorphism. In particular, any function $\Phi$ (surjective or not) induces a poset isomorphism $(P, <) \simeq (\Phi(P), <)$.
        \\

        \item[(ii)] If $(P,<)$ is well-ordered and $\Phi$ is an isomorphism, then $(Q,<)$ is well-ordered.
        \\

    \end{itemize}
\end{proposition}

\begin{proof}
    \begin{itemize}
        \item[(i)] Let $p,p' \in P$ be distinct elements. Without loss of generality, we may assume $p < p'$. But then $\Phi(p) < \Phi(p')$, and in particular $\Phi(p) \neq \Phi(p')$. Therefore $\Phi$ is injective.
        \\

        \item[(ii)] By part (i), $\Phi$ is bijective. We want to show that $\Phi^{-1} : (Q,<) \to (P,<)$ is increasing. Let $q,q' \in Q$ with $q < q'$ and write $p = \Phi(q)$, $p' = \Phi(q')$. In particular, $q \neq q'$, so $p \neq p'$. If $p' < p$, we would have $q' = \Phi(p') < \Phi(p) = q < q'$, a contradiction. Therefore, since $(P,<)$ is linearly ordered, we have $\Phi^{-1}(q) = p < p' = \Phi^{-1}(q')$, so $\Phi^{-1}$ is increasing. The last statement is clear since restricting the codomain of $\Phi$ to its image makes it surjective.
        \\

        \item[(iii)] It follows from Proposition \autoref{extrema-under-poset-morphisms} that for a non-empty subset $T \subseteq Q$, we have 
        \[
            \min T = \Phi(\min \Phi^{-1}(T)),   
        \]
        so $(Q,<)$ is well-ordered.
        \\

    \end{itemize}
\end{proof}

\begin{proposition} \label{posets-as-subposets-of-a-power-set}
    Let $\alpha = (P, <)$ be a well-ordered poset. Then
    \[
        \Phi : P \to \pow P, \quad p \mapsto P_p \defn \{ q \in P \, \mid \, q < p \}
    \]
    is increasing, so that $\Phi : \alpha \to (\Phi(P), \subsetneq)$ is a poset isomorphism. It follows that $(\Phi(P), \subsetneq)$ is well-ordered. If $T \subseteq \Phi(P)$ is a non-empty subset, then
    \[
        \min T = \bigcap_{t \in T} t.    
    \]
\end{proposition}

\begin{proof}
    By Proposition \autoref{morphisms-of-linearly-ordered-posets}, all we have left to prove is that $\min T = \bigcap_{t \in T} t$ for $T \subseteq \Phi(P)$ non-empty. 
    \\
    
    If $t = \min T$, then for all $t' \in T$, we have $t \subsetneq t'$ or $t = t'$, which can be summarized by $t \subseteq t'$. So $\min T$ is characterized by $\min T \in T$ and $\min T \subseteq t$ for all $t \in T$. 
    \\

    By Proposition \autoref{extrema-under-poset-morphisms}, we have the formula $\min T = \Phi(\min \Phi^{-1}(T))$ (in particular, $\min T$ exists), so let $p \defn \min \Phi^{-1}(T)$. We show that $\Phi(p) = \bigcap_{t \in T} t$, which suffices to finish the proof. Since $\alpha$ is linearly ordered, we have $p < \Phi^{-1}(t)$ for all $t \in T \setminus \{\Phi(p)\}$ by definition of $p$, so $\Phi(p) \subsetneq t$ for all such $t$. It follows that $\Phi(p) \subseteq t$ for all $t \in T$, so since $\Phi(p) \in T$, we have $\Phi(p) = \bigcap_{t \in T} t$.
\end{proof}

\begin{remark}
    It is Proposition \autoref{posets-as-subposets-of-a-power-set} that motivated John von Neumann to define the ordinals the way we defined them. Von Neumann ordinals are basically well-ordered subsets where the map $\Phi$ above is the identity, and $\subsetneq$ is equal to $\in$ as a relation on the poset.
\end{remark}

\begin{theorem}[Transfinite induction] \label{transfinite-induction}
    Let $\alpha = (Q, <)$ be a well-ordered poset and let $P$ be a property of elements of $Q$. For $q,q' \in Q$, if whenever $P(q')$ holds for all $q' < q$ implies that $P(q)$ holds, then $P(q)$ is true for all $q \in Q$. 
\end{theorem}

\begin{proof}
    Consider the subset $S \subseteq Q$ of all $q \in Q$ for which $P(q)$ is false, and suppose it is not empty. Since $\alpha$ is well-ordered, let $q = \min S$. By assumption, for all $q' \in Q$ with $q' < q$, $P(q')$ holds because $q' \in Q \setminus S$ since $q = \min S$. But then, by hypothesis, $P(q)$ holds, a contradiction. Therefore $S$ is empty and $P(q)$ holds for all $q \in Q$.
\end{proof}

\begin{proposition} \label{elements-of-ordinals-are-ordinals}
    Let $\alpha$ be an ordinal. Any element $\beta \in \alpha$ is an ordinal.
\end{proposition}

\begin{proof}
    Since $\alpha$ is an ordinal, $\beta \in \alpha$ implies $\beta \subseteq \alpha$. In particular, $\beta$ is well-ordered with the binary relation "$\in$" since $\alpha$ is. We now prove that $\gamma \in \beta$ implies $\gamma \subseteq \beta$. Since $\beta \subseteq \alpha$, we have $\gamma \in \alpha$ and thus $\gamma \subseteq \alpha$. But the binary operation $\in$ is transitive on $\alpha$ since $\alpha$ is a poset, so $\varepsilon \in \gamma$ together with $\gamma \in \beta$ implies $\varepsilon \in \beta$, thus $\gamma \subseteq \beta$. This proves $\beta$ is an ordinal.
\end{proof}

\begin{theorem}[Unicity of the order type] \label{unicity-of-ordinals}
    Let $\alpha$ and $\beta$ be ordinals. If there exists a poset isomorphism $\Phi : \alpha \to \beta$, then $\alpha = \beta$ and $\Phi = \id{\alpha}$. 
\end{theorem}

\begin{proof}
    Let $\Phi : \alpha \to \beta$ be a poset isomorphism. We show by transfinite induction that for all $\gamma \in \alpha$, we have $\gamma = \Phi(\gamma) \in \beta$, which proves that $\alpha \subseteq \beta$ and that $\Phi$ is the inclusion map~; applying the same result to $\Phi^{-1}$ shows $\beta \subseteq \alpha$ and $\Phi^{-1}$ is also an inclusion map, thus proving that $\alpha = \beta$ and $\Phi = \id{\alpha}$. 
    \\
    
    Let $\gamma \in \alpha$. By Proposition \autoref{elements-of-ordinals-are-ordinals}, $\gamma$ is an ordinal. We can consider $\Phi|_{\gamma} : \gamma \to \beta$ since $\gamma \in \alpha$ implies $\gamma \subseteq \alpha$. Suppose that for all $\delta \in \gamma$, we have $\delta = \Phi(\delta)$. It follows that $\gamma \subseteq \Phi(\gamma)$. Conversely, let $\delta' \in \Phi(\gamma)$, so that $\delta' = \Phi(\delta)$ for some $\delta \in \gamma$. This implies $\delta' = \Phi(\delta) = \delta \in \gamma$, so that $\Phi(\gamma) \subseteq \gamma$, which proves the equality $\gamma = \Phi(\gamma)$.
    \\
\end{proof}

\begin{theorem}[Existence of the order type] \label{existence-of-the-order-type}
    Let $(P,<)$ be a well-ordered strict poset. There exists a unique ordinal $\alpha$ with $\alpha \simeq (P,<)$. By Lemma \autoref{unicity-of-ordinals}, $\alpha$ is the unique ordinal with this property, and the isomorphism is also unique~; we call $\alpha$ the \textbf{order type} of $P$. 
    \\
\end{theorem}

\begin{proof}
    Unicity of the ordinal $\alpha$, if it exists, follows by part~(i). For existence, we proceed by transfinite induction on $p \in P$, which is possible since $P$ is well-ordered. For $q \in P$, set $P_q \defn \{ r \in P \, | \, r < q \}$. Suppose that for some $p \in P$, we have the following data for each $q \in P_p$~:
    \\

    \begin{itemize}
        \item[$\bullet$] an ordinal $\alpha_q$
        \\

        \item[$\bullet$] an isomorphism $\Phi_q : P_q \to \alpha_q$
        \\

    \end{itemize}
    and that they satisfy the following:
    \\

    \begin{itemize}
        \item[$\bullet$] the ordinal $\alpha_q$ satisfies $\alpha_q = \{ \alpha_r \, | \, r \in P_q \}$
        \\            
        
        \item[$\bullet$] for $q,r \in P$ with $r < q$, we have $\Phi_q(r) = \alpha_r$.
        \\ 
    \end{itemize}

    We show that 
    \[
        \alpha_p \defn \{ \alpha_q \, | \, q \in P_p \}    
    \]
    is an ordinal and that the function $\Phi_p : P_p \to \alpha_p$ given by $\Phi_p(q) = \alpha_q$ for all $q \in P_p$ is a poset isomorphism. 
    \\

    Note that $\alpha_p$ is linearly ordered since for $q,q' \in P_p$ with $q < q'$, by definition of $\alpha_{q'}$, we have $\alpha_q \in \alpha_{q'}$. Also, $\alpha_q \in \alpha_p$ implies $\alpha_q = \{ \alpha_{q'} \, | \, q' \in P_q \} \subseteq \alpha_p$ since $P_q \subseteq P_p$. It remains to show that $\alpha_p$ is well-ordered to prove it is an ordinal.
    \\
    
    If $T \subseteq \alpha_p$ is a non-empty subset, let $\alpha_q \in T$. If $\min T = \alpha_q$, we have nothing to show. Otherwise, there exists $\alpha_r \in T$ with $\alpha_r < \alpha_q$ (because $\alpha_p$ is linearly ordered), which implies $r < q$ (because $P$ is linearly ordered, $r$ and $q$ have to be distinct since $\alpha_r < \alpha_q$, and $q < r$ implies $\alpha_q < \alpha_r$, a contradiction). We deduce that $\alpha_r \in T \cap \alpha_q \neq \varnothing$, and since $\alpha_q$ is an ordinal with $T \cap \alpha_q \subseteq \alpha_q$, we see that $\min(T \cap \alpha_q)$ exists. Any element of $T \setminus (T \cap \alpha_q)$ is greater than $\min(T \cap \alpha_q)$ since $\alpha_p$ is linearly ordered and $\alpha_r < \alpha_q$ implies $\alpha_r \in \alpha_q$, so $\min T = \min(T \cap \alpha_q)$ exists.
    \\

    It is clear that $\Phi_p : P_p \to \alpha_p$ given by $q \mapsto \alpha_q$ is increasing, since $r < q$ implies $\alpha_r \in \alpha_q$ by definition. By definition of $\alpha_p$, $\Phi_p$ is surjective, so it is an isomorphism. 
    \\

    Now consider 
    \[
        \alpha \defn \{ \alpha_p \, | \, p \in P \}, \quad \Phi : P \to \alpha, \quad p \mapsto \alpha_p.     
    \]
    It is clear that $\alpha$ is linearly ordered since $P$ is. We only used the fact that $\alpha_p$ was linearly ordered to show that it is well-ordered, so the same argument shows that $\alpha$ is also well-ordered. If $\alpha_p \in \alpha$, then $\alpha_p \subseteq \alpha$ because for all $q < p$, $\alpha_q \in \alpha$ as well. Therefore, $\alpha$ is an ordinal. 
    \\

    The map $\Phi$ is increasing since for $q < p$, we have $\alpha_q \in \alpha_p$ by definition of $\alpha_p$. The map $\Phi$ is also surjective by definition of $\Phi$ and $\alpha$, so it is an isomorphism. 
\end{proof}

\begin{lemma} \label{proper-subsets-of-ordinals-are-its-elements}
    Let $\alpha,\beta$ be two ordinals, and suppose that $\alpha \subsetneq \beta$. Then $\alpha \in \beta$. 
\end{lemma}

\begin{proof}
    Let $\gamma \defn \min(\beta \setminus \alpha)$. We show that $\alpha = \gamma$. Note that $\alpha,\gamma \in \beta$, so one of $\alpha \in \gamma$, $\alpha = \gamma$ or $\alpha \ni \gamma$ holds. But $\gamma \in \alpha$ is impossible since $\gamma \in \beta \setminus \alpha$. Since $\alpha \in \gamma$ implies $\alpha \subseteq \gamma$, we now know that $\alpha \subseteq \gamma$.
    \\
    
    Suppose $\delta \in \gamma \setminus \alpha \subseteq \beta \setminus \alpha$. Then $\delta \in \beta \setminus \alpha$ and $\delta \in \gamma$, a contradiction to $\gamma$ being minimal in $\beta \setminus \alpha$. Thus we have the equality $\alpha = \min(\beta \setminus \alpha)$. But this means $\alpha \in \beta \setminus \alpha \subseteq \beta$, e.g. $\alpha \in \beta$.  
\end{proof}

\begin{theorem}[Total ordering of the ordinals] \label{total-ordering-of-the-ordinals}
    Let $\alpha, \beta$ be two ordinals. Then we have the trichotomy $\alpha \in \beta$, $\alpha = \beta$ or $\beta \in \alpha$.
\end{theorem}

\begin{proof}
    % Suppose that $\alpha \not\subseteq \beta$ and $\beta \not\subseteq \alpha$. 
    Let $\gamma \defn \alpha \cap \beta$. It is clear that $\gamma$ is well-ordered since $\alpha$ and $\beta$ both are. If $\delta \in \gamma$, then $\delta \in \gamma \subseteq \alpha$ implies $\delta \subseteq \alpha$ since $\alpha$ is an ordinal, and similarly for $\beta$, implying that $\delta \subseteq \gamma$. This means $\gamma$ is an ordinal. 
    \\

    Now suppose $\gamma \neq \alpha$ and $\gamma \neq \beta$. By definition, that implies $\gamma \subsetneq \alpha$, and similarly $\gamma \subsetneq \beta$, so by Lemma \autoref{proper-subsets-of-ordinals-are-its-elements}, we have $\gamma \in \alpha$ and $\gamma \in \beta$, e.g. $\gamma \in \alpha \cap \beta = \gamma$. This contradicts the Axiom of Regularity, so we must have $\gamma = \alpha$ or $\gamma = \beta$. 
    \\

    Without loss of generality, suppose $\alpha \cap \beta = \alpha$. We have $\alpha = \alpha \cap \beta \subseteq \beta$. If $\alpha = \beta$, we are done. Otherwise, we have $\alpha \subsetneq \beta$, which means $\alpha \in \beta$ by Lemma \autoref{proper-subsets-of-ordinals-are-its-elements}.
\end{proof}

\begin{definition} \label{successor-and-limit-ordinals}
    Let $\alpha$ be an ordinal.
    \\

    \begin{itemize}
        \item[(i)] We say that $\alpha$ is a \textbf{successor ordinal} if $\max \alpha$ exists and $\alpha = S(\max \alpha)$.
        \\
        
        \item[(ii)] We say that $\alpha$ is a \textbf{limit ordinal} if $\alpha \neq \varnothing$ and $\max \alpha$ does not exist.
        \\

    \end{itemize}
\end{definition}

\begin{lemma}[Successor ordinals] \label{successor-ordinals}
    For any ordinal $\alpha$, the set $S(\alpha) \defn \alpha \cup \{\alpha\}$ is also an ordinal.
\end{lemma}

\begin{proof}
    We first show that $S(\alpha)$ is linearly ordered. If $\beta,\gamma \in S(\alpha)$ are distinct, then either both are in $\alpha$ (in which case $\beta \in \gamma$ or $\gamma \in \beta$ because $\alpha$ is linearly ordered), or one of them is equal to $\alpha$~; without loss of generality, assume $\beta = \alpha$. It follows that $\gamma \in S(\alpha) \setminus \{\alpha\} = \alpha = \beta$, as desired.
    \\

    If $T \subseteq S(\alpha)$ is not empty, then either $T \subseteq \alpha$ (in which case $\min T \in \alpha$ exists), or $\alpha \in T$. For all $\beta \in \alpha$, we have $\beta < \alpha$ by definition of the order, so unless $T = \{\alpha\}$ (in which case $\min T = \alpha$), we have $T \setminus \{\alpha\} \subseteq \alpha$ and $\min T = \min (T \setminus \{\alpha\})$ (where the latter minimum exists because $\alpha$ is well-ordered). It follows that $S(\alpha)$ is well-ordered. 
    \\

    If $\beta \in S(\alpha)$, then either $\beta \in \alpha$ (in which case $\beta \subseteq \alpha \subseteq S(\alpha)$) or $\beta = \alpha \subseteq S(\alpha)$. We conclude that $S(\alpha)$ is an ordinal.
    \\
\end{proof}

\begin{proposition} \label{ordinal-is-zero-successor-or-limit}
    An ordinal is either a successor ordinal, a limit ordinal, or zero. 
\end{proposition}

\begin{proof}
    If $\max \alpha$ exists (in which case $\alpha$ is neither zero or a limit ordinal), we have $\max \alpha \in \alpha$, thus $S(\max \alpha) \subseteq \alpha$. By Lemma \autoref{proper-subsets-of-ordinals-are-its-elements}, the inclusion $S(\max \alpha) \subsetneq \alpha$ implies $S(\max \alpha) \in \alpha$, which in turn implies $S(\max \alpha) \le \max \alpha$, a contradiction since $\max \alpha \in S(\max \alpha)$. Therefore $S(\max \alpha) = \alpha$ is a successor ordinal.
\end{proof}